%% Lab Report for EEET2493_labreport_template.tex
%% V1.0
%% 2019/01/16
%% This is the template for a Lab report following an IEEE paper. Modified by Francisco Tovar after Michael Sheel original document.


%% This is a skeleton file demonstrating the use of IEEEtran.cls
%% (requires IEEEtran.cls version 1.8b or later) with an IEEE
%% journal paper.
%%
%% Support sites: 
%% http://www.michaelshell.org/tex/ieeetran/
%% http://www.ctan.org/pkg/ieeetran
%% and
%% http://www.ieee.org/

%%*************************************************************************

%% Legal Notice:
%% This code is offered as-is without any warranty either expressed or
%% implied; without even the implied warranty of MERCHANTABILITY or
%% FITNESS FOR A PARTICULAR PURPOSE! 
%% User assumes all risk.
%% In no event shall the IEEE or any contributor to this code be liable for
%% any damages or losses, including, but not limited to, incidental,
%% consequential, or any other damages, resulting from the use or misuse
%% of any information contained here.
%%
%% All comments are the opinions of their respective authors and are not
%% necessarily endorsed by the IEEE.
%%
%% This work is distributed under the LaTeX Project Public License (LPPL)
%% ( http://www.latex-project.org/ ) version 1.3, and may be freely used,
%% distributed and modified. A copy of the LPPL, version 1.3, is included
%% in the base LaTeX documentation of all distributions of LaTeX released
%% 2003/12/01 or later.
%% Retain all contribution notices and credits.
%% ** Modified files should be clearly indicated as such, including  **
%% ** renaming them and changing author support contact information. **
%%*************************************************************************

\documentclass[journal]{IEEEtran}

% *** CITATION PACKAGES ***
\usepackage[style=ieee]{biblatex} 
\bibliography{analog.bib}    %your file created using JabRef
\usepackage{hyperref}
% *** MATH PACKAGES ***
\usepackage{amsmath}
 \usepackage{multirow}

% *** PDF, URL AND HYPERLINK PACKAGES ***
\usepackage{url}
% correct bad hyphenation here
\hyphenation{op-tical net-works semi-conduc-tor}
\usepackage{graphicx}  %needed to include png, eps figures
\usepackage{float}  % used to fix location of images i.e.\begin{figure}[H]

\begin{document}

% paper title
\title{Laboratorio di elettronica analogica\\ 
%\small{1 gennaio 2020}
}

% author names 
\author{\begin{center}Matteo Barbagiovanni\textsuperscript{1},
        Stefano Barbero\textsuperscript{2},
        Federico Malnati\textsuperscript{3},
        Valerio Pagliarino\textsuperscript{4},
        {\small \\
        \textsuperscript{1}
        matteo.barbagiovanni@edu.unito.it -
        \textsuperscript{2}
        stefano.barbero376@edu.unito.it
        \textsuperscript{3}
        federico.malnati@edu.unito.it -
        \textsuperscript{4}
        valerio.pagliarino@edu.unito.it}
        \end{center}}% <-this % stops a space
        
% The report headers
\markboth{Universita' degli Studi di Torino - C.d.L. Triennale in Fisica - 01/01/19 - A.A. 2021-2022    \quad   \quad \quad \quad   \quad \quad \quad  \quad   \quad \quad \quad   \quad \quad LABORATORIO DI ELETTRONICA \quad \quad }%do not delete next lines
{Shell \MakeLowercase{\textit{et al.}}: Bare Demo of IEEEtran.cls for IEEE Journals}

% make the title area
\maketitle

% As a general rule, do not put math, special symbols or citations
% in the abstract or keywords.
\begin{flushright} Title and abstract: 1 marks. \end{flushright}
\begin{abstract} Provide a summary of the session. What was done, 
what measurements were taken, brief methods, what calculations, brief conclusion.  The Abstract should be approximately 250 words or fewer, italicized, in 10-point Times (or Times Roman.) Please leave two spaces between the Abstract and the heading of your first section.
It should briefly summarize the essence of the paper and address the following areas without using specific subsection titles. Objective: Briefly state the problem or issue addressed, in language accessible to a general scientific audience. Technology or Method: Briefly summarize the technological innovation or method used to address the problem. Results: Provide a brief summary of the results and findings. Conclusions: Give brief concluding remarks on your outcomes. Detailed discussion of these aspects should be provided in the main body of the paper. 
\end{abstract}

\begin{IEEEkeywords}
parole , chiave, parole, chiave
\end{IEEEkeywords}

\section{Materials and Methods. }
\begin{flushright} 1.0 Marks \end{flushright}
\IEEEPARstart{I}{nclude:} one single image that contains 5 subsets of images of the circuits used ( a- inverting amplifier, b - non inverting amplifier, c- High Pass Filter, d- Low Pass Filter, e- Active Band Pass Filter) 0.4 mark.\\[0.1in]
Describe the circuits you used. Refer to image above and describe them briefly (1 paragraph 3-8 lines) 0.3 mark.\\[0.1in]
How circuits were characterized?  What materials, instruments, tools, devices did you used, what variables were measured and over what ranges? (1 paragraph 5-8 lines) 0.3 mark\\

\section{Materials and Methods. }
\begin{flushright} 1.0 Marks \end{flushright}
Include ONE single image that contains 5 subsets of images of the circuits used ( a- inverting amplifier, b - non inverting amplifier, c- High Pass Filter, d- Low Pass Filter, e- Active Band Pass Filter) 0.4 mark.\\[0.1in]
Describe the circuits you used. Refer to image above and describe them briefly (1 paragraph 3-8 lines) 0.3 mark.\\[0.1in]
How circuits were characterized?




\section{Materials and Methods. }
\begin{flushright} 1.0 Marks \end{flushright}
Include ONE single image that contains 5 subsets of images of the circuits used ( a- inverting amplifier, b - non inverting amplifier, c- High Pass Filter, d- Low Pass Filter, e- Active Band Pass Filter) 0.4 mark.\\[0.1in]
Describe the circuits you used. Refer to image above and describe them briefly (1 paragraph 3-8 lines) 0.3 mark.\\[0.1in]
How circuits were characterized? How circuits were characterized? How circuits were characterized? What materials, instruments, tools, devices did you used, what variables were measured and over what ranges? (1 paragraph 5-8 lines) 0.3 mark\\


{\bf Oscilloscope Figures}  
\begin{flushright} 0.4 Marks each figure, including description. \end{flushright}
Figure 2. {\it Inverting Amplifier or Non-Inverting Amplifier}: One Image of the oscilloscope's screen showing the measurement of $v_{out}$/$v_i$ of your choice, Inverting or Non-Inverting configuration.  See Fig.~\ref{fig:oscilloscope} \\
Describe the image and write down the theoretical gain vs experimental gain at other 3 different frequencies, you can refer to the relevant table. (Note: for the table the theoretical wont change, but the experimental may change).\\

\begin{figure}[H]%[!ht]
\begin {center}
\includegraphics[width=0.5\textwidth]{images/BandPass_example01.png}
\caption{Example of figures 1-3. Make sure all numbers in the figure can be read, and that the caption explains the figure.}
\label{fig:oscilloscope}
\end {center}
\end{figure}

Figure 3. {\it Low Pass Filter or High Pass Filter}: One Image of the oscilloscope's screen showing the measurement of $v_{out}$/$v_i$ on either the LPF or HPF.\\
Describe the image and write down the theoretical gain vs experimental gain at 3 different frequencies, you can refer to the relevant table. \\

Figure 4: {\it Active Band Pass Filter}: One Image of the oscilloscope' screen showing the measurement of $v_{out}$/$v_i$ on the active filter (ABPF).\\
Describe the image and write down the theoretical gain vs experimental gain at 3 different frequencies, you can refer to the relevant table.\\


{\bf Tables - 3 Tables}
\begin{flushright} 0.4 Marks each table including description\end{flushright}

\begin{table}[!ht] %[H]
\centering
\caption{High pass filter.}
\begin{tabular}{ccccccc}
Frequency & \multicolumn{2}{l}{Measured Gain} & \multicolumn{2}{l}{Theoretical Gain} \\
Hz   & $v_{out}$/$v_i$    & db     & $v_{out}$/$v_i$  & db           \\ \hline
10   & ??  &  ??           & ??   & ??       \\
100       &                                   &              &                                 &                              \\
etc &  &   &           &      \\                       
etc &  &   &           &      \\  
etc &  &   &           &      

\end{tabular}
\label{table:Exps1}
\end{table}

Table \ref{table:Exps1}, presents the calculated gain from measurements of xxxx, and the calculated from the transfer function of the circuit. Columns: [ 0 Freq, 1 Gain for each frequency tested as $v_{out}$/$v_i$, calculated from measured Vpp values,  2 Gain in db: $20log_{10}(v_{out}/v_{in})$, 3 theoretical gain from the transfer function as $v_{out}$/$v_i$, 4 theoretical gain in db from the transfer function].


\begin{table}[!ht] %[H]
\centering
\caption{Low pass filter.}
\begin{tabular}{ccccccc}
Frequency & \multicolumn{2}{l}{Measured Gain} & \multicolumn{2}{l}{Theoretical Gain} \\
Hz   & $v_{out}$/$v_i$i     & db     & $v_{out}$/$v_i$   & db           \\ \hline
10   & ??  &  ??           & ??   & ??       \\
100       &                                   &              &                                 &                              \\
etc &  &   &           &      \\                       
etc &  &   &           &      \\  
etc &  &   &           &      \\                       
etc &  &   &           &      \\  
etc &  &   &           &      
   
\end{tabular}
\label{table:Exps2}
\end{table}

Table \ref{table:Exps2}, presents ...\\[0.1in]

\begin{table}[!ht] %[H]
\centering
\caption{Active Band Pass Filter.}
\begin{tabular}{ccccccc}
Frequency & \multicolumn{2}{l}{Measured Gain} & \multicolumn{2}{l}{Theoretical Gain} \\
Hz   & $v_{out}$/$v_i$  & db  & $v_{out}$/$v_i$   & db  \\ \hline
10  & ??   & ?? & ??   & ??       \\
100 &  &   &   &   \\
etc &  &   &   &      \\                       
etc &  &   &   &      \\                       
etc &  &   &   &      \\  
etc &  &   &   &      \\  
etc &  &   &   &      

\end{tabular}
\label{table:Exps3}
\end{table}

Table \ref{table:Exps3}, presents ...\\[0.05in]


You can generate a table in a latex format online, then copy-paste it into a latex document. Explore: 
 \href{https://www.tablesgenerator.com}{tablesgenerator.com} \\[0.075in]
 
{\bf Bode and Phase Plots}  
\begin{flushright} 0.4 Marks each figure, including description. \end{flushright}
Figure 5: Bode Plot of either High Pass Filter or Low Pass Filter. See Fig.~\ref{fig:freq_response}. Composed by two lines: experiment (line A) $\&$ theory (line B). Line A: log freq vs log gain from the results from Table I or II, depending on your choice. Line B: 10 calculated theoretical points (log freq vs log gain) using the transfer function of the filter. The bode diagram you have from the soft panel may be included in the appendix only for comparison.\\  

Figure 6: Bode Plot of the Active Band Pass Filter. See Fig.~\ref{fig:freq_response2}. Composed by two lines: experiment (line A) $\&$ theory (line B). Line A: log freq vs log gain from the results from Table III. Line B: 10 calculated theoretical points (log freq vs log gain) using the transfer function of the filter (line B). The bode diagram you have from the soft panel may be included in the appendix only for comparison.\\  

Figure 7: Only one Phase Diagram of any of the three filters tested. It can be a screen shot from NI-Soft Panel or calculated from Excel, its your choice.\\

Figure 8: Optional. Oscilloscope screen-shot waveform of the audio experiment $v_in$ vs $v_{out}$.\\[0.1in]

{\bf Equations to be included when referring to bode plots calculations (discussed on class):}
\begin{itemize}

    \item  Transfer function of the Passive High Pass Filter used.
    \item  Transfer function of the Passive Low Pass Filter used.
    \item  Transfer function of the Active Band Pass Filter used. 

\end{itemize}

\setcounter{figure}{4}

\begin{figure}[H]%[!ht]
\begin {center}
\includegraphics[width=0.50\textwidth]{images/LP_demo.png}
\caption{Example of a frequency response of a filter. Make sure all numbers in the figure can be read, and that the caption explains the figure. This image is missing units in the vertical axis.}
\label{fig:freq_response}
\end {center}
\end{figure}


\begin{figure}[H]%[!ht]
\begin {center}
\includegraphics[width=0.50\textwidth]{images/ABPF_demo.png}
\caption{Example of a frequency response of a filter. Make sure all numbers in the figure can be read, and that the caption explains the figure. This plot is missing Experimental results and legend name for the data line.}
\label{fig:freq_response2}
\end {center}
\end{figure}




\section{Discussion and Summary}
\begin{flushright} 1.0 Mark. \end{flushright}

Summarize your findings. Discuss any interesting result related to the materials used or to any claim from the introduction. Discuss your measurements using engineering terms (accuracy, precision, resolution, etc).  Give technical conclusions. Restate the main objectives and how or to what degree they were achieved. Describe some applications of your results and comment any possible recommended future work.



% if have a single appendix:
%\appendix[Proof of the Zonklar Equations]
% or
%\appendix  % for no appendix heading
% do not use \section anymore after \appendix, only \section*
% is possibly needed

% use appendices with more than one appendix
% then use \section to start each appendix
% you must declare a \section before using any
% \subsection or using \label (\appendices by itself
% starts a section numbered zero.)
%


\appendices
\section{Workout of the transfer functions.}
\begin{flushright} 2.0 Marks. \end{flushright}
Full marks if all workout is presented neatly for the three transfer functions of the filters used. Student clearly demonstrates the origin and development of the concepts used through equations and its workout.
\begin{itemize}

    \item  Transfer function of the Passive High Pass F. (0.5 marks)
    \item Transfer function of the Passive Low Pass F. (0.5 marks)
    \item  Transfer function of the Active Band Pass F. (1.0 marks)

\end{itemize}
% use section* for acknowledgment

% references section

% can use a bibliography generated by BibTeX as a .bbl file
% BibTeX documentation can be easily obtained at:
% http://mirror.ctan.org/biblio/bibtex/contrib/doc/
% The IEEEtran BibTeX style support page is at:
% http://www.michaelshell.org/tex/ieeetran/bibtex/
%\bibliographystyle{IEEEtran}
% argument is your BibTeX string definitions and bibliography database(s)
%\bibliography{IEEEabrv,../bib/paper}
%
% <OR> manually copy in the resultant .bbl file
% set second argument of \begin to the number of references
% (used to reserve space for the reference number labels box)

%use following command to generate the list of cited references

\printbibliography

\section*{References}
\begin{flushright} 0.1 Mark. \end{flushright}

Example of data book:\\[0.1in]
[1] National Operational Amplifiers Databook. Santa Clara: National Semiconductor
Corporation, 1995 Edition, p. I-54. \\[0.1in]
Example of textbook: \\[0.1in]
[2]M. Young, The Technical Writer’s Handbook. Mill Valley, CA: University Science, 1989.\\[0.1in]
Example of scientific journal paper:\\[0.1in]
[3] J.W. Smith, L.S. Alans and D.K. Jones, “An operational amplifier approach to
active cable modeling”, IEEE Transactions on Modeling, vol. 4, no. 2, 1996, pp.
128-132.\\[0.1in]
Example of conference paper proceedings:\\[0.1in]
[4] J.W. Smith, L.S. Alans and D.K. Jones, “Active cable models for lossy
transmission line circuits”, in Proc. 1995 IEEE Modeling Symposium, 1996, pp.
1086-89.\\[0.1in]

Example of Internet web page:\\[0.1in]
[5] Approximate material properties in isotropic materials. Milpitas, CA: Specialty Engineering Associates, Inc. web site: www.ultrasonic.com, downloaded April 20, 2019.  \\[0.1in]

List and number all bibliographical 
references at the end of your paper in {\bf 9 or 10 point} Times, with 10-point interline spacing. When referenced within the text, enclose the citation number in square brackets, for example [1].\\[0.1in]
Use IEEE format. Cite any external work that you used (data sheets, text books, Wikipedia articles, . . . ). If you get a formula from a Wikipedia article, you must cite the article, giving the title, the URL, and the data you accessed the article as a minimum. If you copy a figure, not only must you cite the article you copied from, but you must give explicit figure credit in the caption for the figure: This image copied from . . . . If you modify a figure or base your figure on one that has been published elsewhere, you still need to give credit in the caption: This image adapted from . . . .
% that's all folks
\end{document}


